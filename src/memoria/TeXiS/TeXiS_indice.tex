%---------------------------------------------------------------------
%
%                        TeXiS_indice.tex
%
%---------------------------------------------------------------------
%
% TeXiS_indice.tex
% Copyright 2009 Marco Antonio Gomez-Martin, Pedro Pablo Gomez-Martin
%
% This file belongs to TeXiS, a LaTeX template for writting
% Thesis and other documents. The complete last TeXiS package can
% be obtained from http://gaia.fdi.ucm.es/projects/texis/
%
% This work may be distributed and/or modified under the
% conditions of the LaTeX Project Public License, either version 1.3
% of this license or (at your option) any later version.
% The latest version of this license is in
%   http://www.latex-project.org/lppl.txt
% and version 1.3 or later is part of all distributions of LaTeX
% version 2005/12/01 or later.
%
% This work has the LPPL maintenance status `maintained'.
% 
% The Current Maintainers of this work are Marco Antonio Gomez-Martin
% and Pedro Pablo Gomez-Martin
%
%---------------------------------------------------------------------
%
% Contiene  los  comandos  para  generar  el índice  de  palabras  del
% documento.
%
%---------------------------------------------------------------------
%
% NOTA IMPORTANTE: el  soporte en TeXiS para el  índice de palabras es
% embrionario, y  de hecho  ni siquiera se  describe en el  manual. Se
% proporciona  una infraestructura  básica (sin  terminar)  para ello,
% pero  no ha  sido usada  "en producción".  De hecho,  a pesar  de la
% existencia de  este fichero, *no* se incluye  en Tesis.tex. Consulta
% la documentación en TeXiS_pream.tex para más información.
%
%---------------------------------------------------------------------


% Si se  va a generar  la tabla de  contenidos (el índice  habitual) y
% también vamos a  generar el índice de palabras  (ambas decisiones se
% toman en  función de  la definición  o no de  un par  de constantes,
% puedes consultar modo.tex para más información), entonces metemos en
% la tabla de contenidos una  entrada para marcar la página donde está
% el índice de palabras.

\ifx\generatoc\undefined
\else
   \addcontentsline{toc}{chapter}{\indexname}
\fi


% Generamos el índice
\printindex

% Variable local para emacs, para  que encuentre el fichero maestro de
% compilación y funcionen mejor algunas teclas rápidas de AucTeX

%%%
%%% Local Variables:
%%% mode: latex
%%% TeX-master: "./tesis.tex"
%%% End:
